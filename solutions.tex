\documentclass{article}

\usepackage{fancyhdr}
\usepackage{extramarks}
\usepackage{amsmath}
\usepackage{amsthm}
\usepackage{amssymb}
\usepackage{amsfonts}
\usepackage{tikz}
\usepackage{physics}
\usepackage[plain]{algorithm}
\usepackage{algpseudocode}
\usepackage[makeroom]{cancel}
\usetikzlibrary{automata,positioning}

%
% Basic Document Settings
%

\topmargin=-0.45in
\evensidemargin=0in
\oddsidemargin=0in
\textwidth=6.5in
\textheight=9.0in
\headsep=0.25in

\linespread{1.1}

\pagestyle{fancy}
\lhead{\hmwkAuthorName}
\chead{\hmwkClass\ : \hmwkTitle}
\rhead{\firstxmark}
\lfoot{\lastxmark}
\cfoot{\thepage}

\renewcommand\headrulewidth{0.4pt}
\renewcommand\footrulewidth{0.4pt}

\setlength\parindent{0pt}

%
% Create Problem Sections
%
\newcommand{\be}{\begin{equation}}
\newcommand{\ee}{\end{equation}}
\newcommand{\bes}{\begin{equation*}}
\newcommand{\ees}{\end{equation*}}
\newcommand{\bea}{\begin{flalign*}}
\newcommand{\eea}{\end{flalign*}}


\newcommand{\enterProblemHeader}[1]{
    \nobreak\extramarks{}{Problem \arabic{#1} continued on next page\ldots}\nobreak{}
    \nobreak\extramarks{Problem \arabic{#1} (continued)}{Problem \arabic{#1} continued on next page\ldots}\nobreak{}
}

\newcommand{\exitProblemHeader}[1]{
    \nobreak\extramarks{Problem \arabic{#1} (continued)}{Problem \arabic{#1} continued on next page\ldots}\nobreak{}
    \stepcounter{#1}
    \nobreak\extramarks{Problem \arabic{#1}}{}\nobreak{}
}

\setcounter{secnumdepth}{0}
\newcounter{partCounter}
\newcounter{homeworkProblemCounter}
\setcounter{homeworkProblemCounter}{1}
\nobreak\extramarks{Problem \arabic{homeworkProblemCounter}}{}\nobreak{}

%
% Homework Problem Environment
%
% This environment takes an optional argument. When given, it will adjust the
% problem counter. This is useful for when the problems given for your
% assignment aren't sequential. See the last 3 problems of this template for an
% example.
%
\newenvironment{homeworkProblem}[1][-1]{
    \ifnum#1>0
        \setcounter{homeworkProblemCounter}{#1}
    \fi
    \subsection{Problem \arabic{homeworkProblemCounter}}
    \setcounter{partCounter}{1}
    \enterProblemHeader{homeworkProblemCounter}
}{
    \exitProblemHeader{homeworkProblemCounter}
}

%
% Homework Details
%   - Title
%   - Due date
%   - Class
%   - Section/Time
%   - Instructor
%   - Author
%

\newcommand{\hmwkTitle}{Homework\ \#2}
\newcommand{\hmwkDueDate}{February 12, 2014}
\newcommand{\hmwkClass}{Calculus}
\newcommand{\hmwkClassTime}{}
\newcommand{\hmwkClassInstructor}{Professor Isaac Newton}
\newcommand{\hmwkAuthorName}{\textbf{Aditya Vijaykumar}}

%
% Title Page
%

\title{
    %\vspace{2in}
    \textmd{\textbf{\hmwkClass:\ \hmwkTitle}}\\
    \normalsize\vspace{0.1in}\small{\hmwkDueDate\ }\\
%    \vspace{3in}
}

\author{\hmwkAuthorName}
\date{}

\renewcommand{\part}[1]{\textbf{\large Part \Alph{partCounter}}\stepcounter{partCounter}\\}

%
% Various Helper Commands
%

% Useful for algorithms
\newcommand{\alg}[1]{\textsc{\bfseries \footnotesize #1}}

% For derivatives
\newcommand{\deriv}[1]{\frac{\mathrm{d}}{\mathrm{d}x} (#1)}

% For partial derivatives
\newcommand{\pderiv}[2]{\frac{\partial}{\partial #1} (#2)}

% Integral dx
\newcommand{\dx}{\mathrm{d}x}

% Alias for the Solution section header
\newcommand{\solution}{\textbf{\large Solution}}

% Probability commands: Expectation, Variance, Covariance, Bias
\newcommand{\E}{\mathrm{E}}
\newcommand{\Var}{\mathrm{Var}}
\newcommand{\Cov}{\mathrm{Cov}}
\newcommand{\Bias}{\mathrm{Bias}}

\begin{document}

\maketitle

\section{\underline{Numerical Models}}

%\pagebreak

\begin{homeworkProblem}[1]
    \textbf{The Conservation Laws}
    
    \begin{enumerate}
    	\item The problem is to find orthonormal tetrad $ e_{(m)}^a $ \textit{ie},
    	\begin{equation}\label{key}
    	g^{ab} =  e_{(m)}^a  e_{(n)}^b \delta^{mn}
    	\end{equation}
    	for Minkowski spacetime in both Cartesian and spherical coordinates.
    \end{enumerate}
\end{homeworkProblem}



\begin{homeworkProblem}[2]
	\textbf{The perfect fluid}
	\begin{enumerate}
	\item We have, 
	\begin{equation}\label{key}
		T_{ab} = \rho_0 h u_a u_b +  p g_{ab}
		\end{equation}
	A generic tensor $ T_{ab} $ in four dimensions would have 16 independent degrees of freedom. But we notice that $ T_{ab} $ here is symmetric under the exchange $ a \rightarrow b $ and hence this leaves us with 10 independent degrees of freedom. Furthermore, we have the constraint $ \grad_a T^{ab} = 0 $ which gives us 4 additional constraints reducing the independent degrees of freedom to 6. Then, using the the equation of state $ p = p(\rho_0 , \epsilon) $, we can eliminate one more degree of freedom, and hence $ T_{ab} $ has five degrees of freedom.
	\end{enumerate}
\end{homeworkProblem}


\begin{homeworkProblem}[3]
	\textbf{Action Principles}
	
\end{homeworkProblem}

\newpage
\section{\underline{Numerical Theory}}
\begin{homeworkProblem}[1]
	\textbf{Finite Differencing}
	\begin{enumerate}
		\item Using Taylor Expansion,
		\begin{align}\label{key}
		 \eval{\dv{f}{x}}_{x=x_0} & =  \dfrac{\cancel{f(x_0) }+ f'(x_0) \Delta x  + \order{\Delta x}^2 - \cancel{f(x_0) }}{\Delta x } \\
		 &= f'(x_0) + \order{\Delta x}
		\end{align}
		
		\item \textit{There is a typo in the question : $ f(x_0 - \Delta x) $ and not $ f(x_0 - \Delta) $} 
		
		Again, using Taylor expansion,
		\begin{align}\label{key}
		\eval{\dv{f}{x}}_{x=x_0} & =  \dfrac{\cancel{f(x_0) }+ f'(x_0) \Delta x + \cancel{\dfrac{f''(x_0) (\Delta x)^2}{2}}  + \order{\Delta x}^3 - \cancel{f(x_0)} + f'(x_0 ) \Delta x - \cancel{\dfrac{f''(x_0) (\Delta x)^2}{2}} + \order{\Delta x}^3 }{2\Delta x } \\
		&= f'(x_0) + \order{\Delta x}^2
		\end{align}
		
		
		\item Let's Taylor expand the two sides,
		\begin{align}\label{key}
		\dfrac{y(x_0 + \Delta x) - y(x_0)}{\Delta x} + \order{\Delta x} &=  F(x_0, y(x_0)) +\order{\Delta x}\\
		\implies y(x_0 + \Delta x) &= y(x_0) + \Delta x F(x_0, y(x_0))  + \order{\Delta x}^2
		\end{align}
		
		Where, in the second step, we have multiplied by $ \Delta x $ and rearranged the terms.
		
		\item 
		
		\item The advection equation for $ q = q(x,t) $ with constant velocity $ v $ is,
		\begin{equation}\label{advection}
		\partial_t q - v \partial_x q = 0 
		\end{equation}
		We can expand $ q $ about $ x_0 $ and $ t_0 $ as follows,
		\begin{equation}\label{key}
		\partial_t q = \dfrac{q(x_0,t_0 + \Delta t) - q(x_0, t_0)}{\Delta t} + \order{\Delta t, \Delta x} \qq{and} \partial_x q = \dfrac{q(x_0 + \Delta x,t_0  ) - q(x_0, t_0)}{\Delta x} +  \order{\Delta t, \Delta x}
		\end{equation}
		Now substituting these into Eq. \ref{advection},
		\begin{align}\label{key}
		\dfrac{q(x_0,t_0 + \Delta t) - q(x_0, t_0)}{\Delta t} - v\dfrac{q(x_0 + \Delta x,t_0  ) - q(x_0, t_0)}{\Delta x} + \order{\Delta t, \Delta x}  = 0 \\
		\implies q(x_0,t_0 + \Delta t) =  q(x_0, t_0) + v\Delta t\dfrac{q(x_0 + \Delta x,t_0  ) - q(x_0, t_0)}{\Delta x} + \order{\Delta t, \Delta x}
		\end{align}
		\end{enumerate}
	
\end{homeworkProblem}


\begin{homeworkProblem}[2]
	\textbf{Modified Equation}
	\begin{enumerate}
		\item The first order upwind scheme is given by,
		\begin{equation}\label{key}
		q(x_0, t_0 + \Delta t) = q(x_0, t_0) + \dfrac{v \Delta t}{\Delta x}\qty[q(x_0 - \Delta x,t_0 ) - q(x_0, t_0)]
		\end{equation}
		We Taylor-expand all terms that have $ \Delta $'s in them,
		\begin{align}\label{key}
		\cancel{q(x_0, t_0)}  +  \partial_t q \ \Delta t &= \cancel{q(x_0, t_0)} + \dfrac{v \Delta t}{\Delta x}\qty[\cancel{q(x_0,t_0 ) }- \partial_x q \ \Delta x  + \partial_{xx} q \ \dfrac{\Delta x^2}{2}-\cancel{ q(x_0, t_0)}]\\
		\partial_t q   &= \dfrac{v }{\Delta x}\qty[- \partial_x q \ \Delta x  + \partial_{xx} q \ \dfrac{\Delta x^2}{2}] \\
		\partial_t q + v \partial_x q &=\dfrac{v \Delta x}{2} \qty(1 - \dfrac{v \Delta t}{\Delta x}) \partial_{xx} q,
		\end{align}
		where is going from the first to the second step, we have equated the coefficients of $ \Delta t $ on both sides of the equation. The term $  \beta =  \dfrac{v \Delta x}{2} \qty(1 - \dfrac{v \Delta t}{\Delta x}) $ acts as an effective numerical viscosity, and hence  \textit{diffusing }the numerics leading to numerical errors.
		
		\item The violation of the CFL condition leads to the viscosity $ \beta $ turning negative, hence the \textcolor{red}{solution will grow exponentially and will blow up}. 
		
		\item Substituting $ f(x-vt) = q $ in the advection equation and using the chain rule of differentiation,
		\begin{equation}\label{key}
		-v \ q'(\eta) + v q'(\eta) = 0
		\end{equation}
		Hence, all $ q(\eta) = f(x-vt) $ is a solution of the advection equation.
		
		\item  Substituting $ f(x/t) = q(\xi) $ in the advection equation and using the chain rule of differentiation,
		\begin{equation}\label{key}
		-\dfrac{x}{t^2} q'(\xi) + \dfrac{v}{t} q'(\xi) = 0 \implies \dfrac{x}{t} = \xi = v
		\end{equation}
		
		\item Substituting $ f\qty(\dfrac{x-vt}{t^\alpha}) = q(\eta) $ in the modified equation and using the chain rule of differentiation,
		\begin{align}\label{key}
		\qty(-\dfrac{\alpha x}{t^{\alpha + 1}} + \dfrac{v (\alpha - 1)}{t^\alpha} ) q'(\eta)  + \dfrac{v}{t^\alpha} q'(\eta) = \beta \dfrac{v^2}{t^{2\alpha}} q''(\eta) \\
		\implies - \alpha \dfrac{x-vt}{t^{\alpha + 1}} q' = \beta \dfrac{v^2}{t^{2\alpha}} q'' \implies  \beta \dfrac{v^2}{t^{2\alpha}} q'' + \alpha \dfrac{\eta}{t} q' = 0
		\end{align}
		
		\textcolor{red}{Write in terms of error functions}
		 
		\item  ifhas
	\end{enumerate}
\end{homeworkProblem}


\begin{homeworkProblem}
	\begin{enumerate}
		\item In Part 3 of Problem 2, we proved that $ q = f(x - vt) $ is always a solution to the advection equation. Hence $ q (x,t) = \exp\qty[i \gamma (x-vt)] $ is a solution to the advection equation. As we are given the initial data $ q(x,t) = \exp(i \ell x ) $, we conclude that $ \gamma = \ell $ for this problem. 
		
		\item Substituting the form of the solution into the second order difference formula,
		\begin{align}\label{partialq2cd}
		\partial_x q &= \dfrac{\exp[i \ell (x_k - v t)]}{2 \Delta x} \qty{\exp(i \ell \Delta x) - \exp( -i \ell \Delta x)} \\
		&=  \dfrac{\exp[i \ell (x_k - v t)]}{\Delta x} i \sin(  \ell \Delta x)
		\end{align} 
		
		\item Substituting $ q_e $ and $ q_{m,\Delta x} $,
		\begin{equation}
		\dfrac{q_e(x,t) - q_{m,\Delta x} (x,t)}{q_e(x,t)} = 1 - \exp[i \ell (v - v_m(\ell))T]
		\end{equation}
		Taking the limit $ \Delta x \rightarrow 0 $ and retaining only upto next-to-leading order,
		\begin{align}
		\dfrac{q_e(x,t) - q_{m,\Delta x} (x,t)}{q_e(x,t)} &\approx - i \sin[ \ell (v - v_m(l))T] \\
		\therefore e_m(\ell) =  \abs{\dfrac{q_e(x,t) - q_{m,\Delta x} (x,t)}{q_e(x,t)}} &\approx \ell (v - v_m(l))T
		\end{align}
		
		\item Substituting the expression for $ v_m $ in the result of the previous part,
		\begin{equation}\label{key}
		e_m(l) = \ell v  T \qty(1 - \dfrac{\sin(\ell \Delta x)}{\ell \Delta x}) \approx  \ell v  T \dfrac{(l\Delta x)^2}{6}
		\end{equation}
	\end{enumerate}

\end{homeworkProblem}

\end{document}
