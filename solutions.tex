\documentclass{article}

\usepackage{fancyhdr}
\usepackage{extramarks}
\usepackage{amsmath}
\usepackage{amsthm}
\usepackage{amssymb}
\usepackage{amsfonts}
\usepackage{tikz}
\usepackage{physics}
\usepackage[plain]{algorithm}
\usepackage{algpseudocode}
\usepackage[makeroom]{cancel}
\usetikzlibrary{automata,positioning}

%
% Basic Document Settings
%

\topmargin=-0.45in
\evensidemargin=0in
\oddsidemargin=0in
\textwidth=6.5in
\textheight=9.0in
\headsep=0.25in

\linespread{1.1}

\pagestyle{fancy}
\lhead{\hmwkAuthorName}
\chead{\hmwkClass\  \hmwkTitle}
\rhead{\firstxmark}
\lfoot{\lastxmark}
\cfoot{\thepage}

\renewcommand\headrulewidth{0.4pt}
\renewcommand\footrulewidth{0.4pt}

\setlength\parindent{0pt}

%
% Create Problem Sections
%
\newcommand{\be}{\begin{equation}}
\newcommand{\ee}{\end{equation}}
\newcommand{\bes}{\begin{equation*}}
\newcommand{\ees}{\end{equation*}}
\newcommand{\bea}{\begin{flalign*}}
\newcommand{\eea}{\end{flalign*}}


\newcommand{\enterProblemHeader}[1]{
    \nobreak\extramarks{}{Problem \arabic{#1} continued on next page\ldots}\nobreak{}
    \nobreak\extramarks{Problem \arabic{#1} (continued)}{Problem \arabic{#1} continued on next page\ldots}\nobreak{}
}

\newcommand{\exitProblemHeader}[1]{
    \nobreak\extramarks{Problem \arabic{#1} (continued)}{Problem \arabic{#1} continued on next page\ldots}\nobreak{}
    \stepcounter{#1}
    \nobreak\extramarks{Problem \arabic{#1}}{}\nobreak{}
}

\setcounter{secnumdepth}{0}
\newcounter{partCounter}
\newcounter{homeworkProblemCounter}
\setcounter{homeworkProblemCounter}{1}
\nobreak\extramarks{Problem \arabic{homeworkProblemCounter}}{}\nobreak{}

%
% Homework Problem Environment
%
% This environment takes an optional argument. When given, it will adjust the
% problem counter. This is useful for when the problems given for your
% assignment aren't sequential. See the last 3 problems of this template for an
% example.
%
\newenvironment{homeworkProblem}[1][-1]{
    \ifnum#1>0
        \setcounter{homeworkProblemCounter}{#1}
    \fi
    \subsection{Problem \arabic{homeworkProblemCounter}}
    \setcounter{partCounter}{1}
    \enterProblemHeader{homeworkProblemCounter}
}{
    \exitProblemHeader{homeworkProblemCounter}
}

%
% Homework Details
%   - Title
%   - Due date
%   - Class
%   - Section/Time
%   - Instructor
%   - Author
%

\newcommand{\hmwkTitle}{}
\newcommand{\hmwkDueDate}{ICTS Summer School on Gravitational-wave Astronomy}
\newcommand{\hmwkClass}{Numerical Hydrodynamics - Solutions}
\newcommand{\hmwkClassTime}{}
\newcommand{\hmwkClassInstructor}{Professor Isaac Newton}
\newcommand{\hmwkAuthorName}{\textbf{Aditya Vijaykumar}}

%
% Title Page
%

\title{
    %\vspace{2in}
    \textmd{\textbf{\hmwkClass\ \hmwkTitle}}\\
    \normalsize\vspace{0.1in}\small{\hmwkDueDate\ }\\
%    \vspace{3in}
}

\author{\hmwkAuthorName}
\date{}

\renewcommand{\part}[1]{\textbf{\large Part \Alph{partCounter}}\stepcounter{partCounter}\\}

%
% Various Helper Commands
%

% Useful for algorithms
\newcommand{\alg}[1]{\textsc{\bfseries \footnotesize #1}}

% For derivatives
\newcommand{\deriv}[1]{\frac{\mathrm{d}}{\mathrm{d}x} (#1)}

% For partial derivatives
\newcommand{\pderiv}[2]{\frac{\partial}{\partial #1} (#2)}

% Integral dx
\newcommand{\dx}{\mathrm{d}x}

% Alias for the Solution section header
\newcommand{\solution}{\textbf{\large Solution}}

% Probability commands: Expectation, Variance, Covariance, Bias
\newcommand{\E}{\mathrm{E}}
\newcommand{\Var}{\mathrm{Var}}
\newcommand{\Cov}{\mathrm{Cov}}
\newcommand{\Bias}{\mathrm{Bias}}

\begin{document}

\maketitle


\section{\underline{Numerical Theory}}
\begin{homeworkProblem}[1]
	\textbf{Finite Differencing}
	\begin{enumerate}
		\item Using Taylor Expansion,
		\begin{align}\label{key}
		 \eval{\dv{f}{x}}_{x=x_0} & =  \dfrac{\cancel{f(x_0) }+ f'(x_0) \Delta x  + \order{\Delta x}^2 - \cancel{f(x_0) }}{\Delta x } \\
		 &= f'(x_0) + \order{\Delta x}
		\end{align}
		
		\item \textit{There is a typo in the question : $ f(x_0 - \Delta x) $ and not $ f(x_0 - \Delta) $} 
		
		Again, using Taylor expansion,
		\begin{align}\label{key}
		\eval{\dv{f}{x}}_{x=x_0} & =  \dfrac{\cancel{f(x_0) }+ f'(x_0) \Delta x + \cancel{\dfrac{f''(x_0) (\Delta x)^2}{2}}  + \order{\Delta x}^3 - \cancel{f(x_0)} + f'(x_0 ) \Delta x - \cancel{\dfrac{f''(x_0) (\Delta x)^2}{2}} + \order{\Delta x}^3 }{2\Delta x } \\
		&= f'(x_0) + \order{\Delta x}^2
		\end{align}
		
		
		\item Let's Taylor expand the two sides,
		\begin{align}\label{key}
		\dfrac{y(x_0 + \Delta x) - y(x_0)}{\Delta x} + \order{\Delta x} &=  F(x_0, y(x_0)) +\order{\Delta x}\\
		\implies y(x_0 + \Delta x) &= y(x_0) + \Delta x F(x_0, y(x_0))  + \order{\Delta x}^2
		\end{align}
		
		Where, in the second step, we have multiplied by $ \Delta x $ and rearranged the terms.
		
		\item Integrating the differential equation between $ x_0 $ and $ x_0 + \Delta x $, we get,
		\begin{equation}\label{key}
		y(x_0 + \Delta x) - y(x_0) = \int_{x_0}^{x_0 + \Delta x} F(x,y) \dd x
		\end{equation}
		In the infinitesimal limit, we assume that $ F(x,y) $ within the integration limits $ \approx F(x_0, y(x_0)) $. Hence,
		\begin{equation}\label{key}
		y(x_0 + \Delta x) - y(x_0) = F(x,y) \Delta x \dd x \qq{,}
		\end{equation}
		which is the Euler method.
		
		\item The advection equation for $ q = q(x,t) $ with constant velocity $ v $ is,
		\begin{equation}\label{advection}
		\partial_t q - v \partial_x q = 0 
		\end{equation}
		We can expand $ q $ about $ x_0 $ and $ t_0 $ as follows,
		\begin{equation}\label{key}
		\partial_t q = \dfrac{q(x_0,t_0 + \Delta t) - q(x_0, t_0)}{\Delta t} + \order{\Delta t, \Delta x} \qq{and} \partial_x q = \dfrac{q(x_0 + \Delta x,t_0  ) - q(x_0, t_0)}{\Delta x} +  \order{\Delta t, \Delta x}
		\end{equation}
		Now substituting these into Eq. \ref{advection},
		\begin{align}\label{key}
		\dfrac{q(x_0,t_0 + \Delta t) - q(x_0, t_0)}{\Delta t} - v\dfrac{q(x_0 + \Delta x,t_0  ) - q(x_0, t_0)}{\Delta x} + \order{\Delta t, \Delta x}  = 0 \\
		\implies q(x_0,t_0 + \Delta t) =  q(x_0, t_0) + v\Delta t\dfrac{q(x_0 + \Delta x,t_0  ) - q(x_0, t_0)}{\Delta x} + \order{\Delta t, \Delta x}
		\end{align}
		
		\item 
		\end{enumerate}
	
\end{homeworkProblem}


\begin{homeworkProblem}[2]
	\textbf{Modified Equation}
	\begin{enumerate}
		\item The first order upwind scheme is given by,
		\begin{equation}\label{key}
		q(x_0, t_0 + \Delta t) = q(x_0, t_0) + \dfrac{v \Delta t}{\Delta x}\qty[q(x_0 - \Delta x,t_0 ) - q(x_0, t_0)]
		\end{equation}
		We Taylor-expand all terms that have $ \Delta $'s in them,
		\begin{align}\label{key}
		\cancel{q(x_0, t_0)}  +  \partial_t q \ \Delta t &= \cancel{q(x_0, t_0)} + \dfrac{v \Delta t}{\Delta x}\qty[\cancel{q(x_0,t_0 ) }- \partial_x q \ \Delta x  + \partial_{xx} q \ \dfrac{\Delta x^2}{2}-\cancel{ q(x_0, t_0)}]\\
		\partial_t q   &= \dfrac{v }{\Delta x}\qty[- \partial_x q \ \Delta x  + \partial_{xx} q \ \dfrac{\Delta x^2}{2}] \\
		\partial_t q + v \partial_x q &=\dfrac{v \Delta x}{2} \qty(1 - \dfrac{v \Delta t}{\Delta x}) \partial_{xx} q,
		\end{align}
		where is going from the first to the second step, we have equated the coefficients of $ \Delta t $ on both sides of the equation. The term $  \beta =  \dfrac{v \Delta x}{2} \qty(1 - \dfrac{v \Delta t}{\Delta x}) $ acts as an effective numerical viscosity, and hence  \textit{diffusing }the numerics leading to numerical errors.
		
		\item The violation of the CFL condition leads to the viscosity $ \beta $ turning negative. In a physical situation, this would be lead to the \textit{opposite} of diffusion as we know it; this means that there will be energy added into the system, and smooth features in the solution will eventually go to sharp features and giving infinities.
		
		\item Substituting $ f(x-vt) = q $ in the advection equation and using the chain rule of differentiation,
		\begin{equation}\label{key}
		-v \ q'(\eta) + v q'(\eta) = 0
		\end{equation}
		Hence, all $ q(\eta) = f(x-vt) $ is a solution of the advection equation.
		
		\item  Substituting $ f(x/t) = q(\xi) $ in the advection equation and using the chain rule of differentiation,
		\begin{equation}\label{key}
		-\dfrac{x}{t^2} q'(\xi) + \dfrac{v}{t} q'(\xi) = 0 \implies \dfrac{x}{t} = \xi = v
		\end{equation}
		
		\item Substituting $ f\qty(\dfrac{x-vt}{t^\alpha}) = q(\eta) $ in the modified equation and using the chain rule of differentiation,
		\begin{align}\label{key}
		\qty(-\dfrac{\alpha x}{t^{\alpha + 1}} + \dfrac{v (\alpha - 1)}{t^\alpha} ) q'(\eta)  + \dfrac{v}{t^\alpha} q'(\eta) = \beta \dfrac{v^2}{t^{2\alpha}} q''(\eta) \\
		\implies - \alpha \dfrac{x-vt}{t^{\alpha + 1}} q' = \beta \dfrac{v^2}{t^{2\alpha}} q'' \implies  \beta \dfrac{v^2}{t^{2\alpha}} q'' + \alpha \dfrac{\eta}{t} q' = 0
		\end{align}

		For $  \alpha = 1/2$, the equation,
		\begin{equation}\label{key}
		\beta\dfrac{v^2}{t} q'' + \dfrac{\eta}{2t} q' = 0 \implies \beta{v^2} q'' + \dfrac{\eta}{2} q' = 0  \qq{as}  t>0
		\end{equation}
		Assuming $ q'(\eta)  = h(\eta) $,
		\begin{align}\label{key}
		\beta{v^2} h' + \dfrac{\eta}{2} h = 0  &\implies h(\eta) = q'(\eta) = C_1 \exp(-\dfrac{\eta}{2\beta v^2}) \\
		& \implies q(\eta ) = C_1 \int \dd \eta \exp(-\dfrac{\eta}{2\beta v^2}) \qq{.} 
		\end{align}
		The Gaussian integral is in fact the error function.
		\item  
		\item 
	\end{enumerate}
\end{homeworkProblem}


\begin{homeworkProblem}
	\textbf{Phase errors and neutron stars}
	\begin{enumerate}
		\item In Part 3 of Problem 2, we proved that $ q = f(x - vt) $ is always a solution to the advection equation. Hence $ q (x,t) = \exp\qty[i \gamma (x-vt)] $ is a solution to the advection equation. As we are given the initial data $ q(x,t) = \exp(i \ell x ) $, we conclude that $ \gamma = \ell $ for this problem. 
		
		\item Substituting the form of the solution into the second order difference formula,
		\begin{align}\label{partialq2cd}
		\partial_x q &= \dfrac{\exp[i \ell (x_k - v t)]}{2 \Delta x} \qty{\exp(i \ell \Delta x) - \exp( -i \ell \Delta x)} \\
		&=  \dfrac{\exp[i \ell (x_k - v t)]}{\Delta x} i \sin(  \ell \Delta x)
		\end{align} 
		
		\item Substituting $ q_e $ and $ q_{m,\Delta x} $,
		\begin{equation}
		\dfrac{q_e(x,t) - q_{m,\Delta x} (x,t)}{q_e(x,t)} = 1 - \exp[i \ell (v - v_m(\ell))T]
		\end{equation}
		Taking the limit $ \Delta x \rightarrow 0 $ and retaining only upto next-to-leading order,
		\begin{align}
		\dfrac{q_e(x,t) - q_{m,\Delta x} (x,t)}{q_e(x,t)} &\approx - i \sin[ \ell (v - v_m(l))T] \\
		\therefore e_m(\ell) =  \abs{\dfrac{q_e(x,t) - q_{m,\Delta x} (x,t)}{q_e(x,t)}} &\approx \ell (v - v_m(l))T
		\end{align}
		
		\item Substituting the expression for $ v_m $ in the result of the previous part,
		\begin{equation}\label{key}
		e_{\mathrm{2cd}}(l) = \ell v  T \qty(1 - \dfrac{\sin(\ell \Delta x)}{\ell \Delta x}) \approx  \ell v  T \dfrac{(l\Delta x)^2}{6}
		\end{equation}
		
		\item  Given $ p = \dfrac{2\pi}{\ell \Delta x} $ and $ \nu = \dfrac{\ell v T}{2 \pi} $ , so $ \ell \Delta x = \dfrac{2 \pi}{p} $ and $ T = \dfrac{2\pi \nu}{\ell v} $. Substituting these in the expression for $ e_{\mathrm{2cd}}(l) $, we get the required result,
		\begin{equation}\label{key}
		e_{\mathrm{2cd}} =  \dfrac{\pi \nu}{3} \qty(\dfrac{2\pi}{p})^2 \qq{.}
		\end{equation}
		We also get the minimum points per wavelength for a required phase error to be,
		\begin{equation}\label{key}
		p_{\mathrm{2cd}} = 2 \pi \sqrt{\dfrac{\nu \pi}{3 e_{\mathrm{2cd}}}}
		\end{equation}
		\item  Substituting $ v =  $ for $ \ell = 2 $ (corresponding to the fundamental mode)
		
		\item Subsituting $ e_\mathrm{2cd} = 0.01 $ (1\% error) and $ \nu \approx 30  $ in the expression for $ p_\mathrm{2cd} $, we get $ p_\mathrm{2cd} \approx 350 .$ 
		
		\item $ \Delta x = \dfrac{2\pi}{\ell p} $
		
		\item 
		
		\item 
		
	\end{enumerate}

\end{homeworkProblem}

\begin{homeworkProblem}[4]
	\textbf{Vacuum part 1}
\end{homeworkProblem}

\begin{homeworkProblem}[5]
	\textbf{Vacuum part 2}
	\begin{enumerate}
		\item The conservation law is,
		\begin{equation}\label{key}
		\pdv{\vb{q}}{t}+ \pdv{ \vb{f}(\vb{q})}{x} = 0 \implies \pdv{\vb{q}}{t}+ \pdv{ \vb{f}(\vb{q})}{\vb{q}} \pdv{\vb{q}}{x} = 0 \implies \pdv{\vb{q}}{t}+ J \pdv{\vb{q}}{x} = 0 \qq{.}
		\end{equation}
		where $ J = \pdv{ \vb{f}(\vb{q})}{\vb{q}}  $. Substituting $ S^{-1} \vb{w} =  \vb{q} $ subject to $ S J S^{-1} = \Lambda $, where $ \Lambda $ is the diagonal matrix containing the eigenvalues of $ J $,
		\begin{equation}\label{key}
		S^{-1}\pdv{\vb{w}}{t}+ J S^{-1}\pdv{\vb{w}}{x} = 0  \implies \pdv{\vb{w}}{t}+ S J S^{-1}\pdv{\vb{w}}{x} = 0  \implies \pdv{\vb{w}}{t}+ \Lambda\pdv{\vb{w}}{x} = 0 
		\end{equation}
		
		\item 
	\end{enumerate}
\end{homeworkProblem}
 
\begin{homeworkProblem}[6]
	\textbf{Well Balancing}
	\begin{enumerate}
		\item This can be easily seen by substituting the form $ q = Ce^x $ in the advection equation with source.
		
		\item The advection equation with source is given by,
		\begin{equation}\label{key}
		\partial_t q + \partial_x q = q
		\end{equation}
		Using forward differencing for $ \partial_t q  $ and backward differencing for $ \partial_x q $ we have,
		\begin{align}\label{key}
		\dfrac{q(x_0, t_0 + \Delta t) - q(x_0, t_0)}{\Delta t} + \dfrac{q(x_0- \Delta x, t_0) - q(x_0, t_0)}{- \Delta x} = q(x_0, t_0) \\
		\implies q(x_0, t_0 + \Delta t) = q(x_0.t_0) - \dfrac{\Delta t}{\Delta x} \qty(q(x_0,t_0) - q(x_0 - \Delta x, t_0)) + \Delta t q(x_0,t_0)
		\end{align}
		
		\item 
	\end{enumerate}

\end{homeworkProblem}


\begin{homeworkProblem}[7]
	\textbf{Shocks}
	\begin{enumerate}
		\item 
		
		\item 
		\begin{equation}\label{key}
		\partial_t (q^n) + \dfrac{n}{n+1} \partial_x (q^{n+1} ) = 0 \implies n q^{n-1} \partial_t q + n  q^n \partial_x q = 0 \implies \partial_t q + q \partial_q q = 0 \ ,
		\end{equation}
		which is the Burgers equation.
		
		\item ds
	\end{enumerate}


\end{homeworkProblem}

\begin{homeworkProblem}[8]
	\textbf{Telescoping}
	\begin{enumerate}
		\item The first order upwind scheme for  $ \partial_t q + v \partial_x q = 0 $ is given by,
		\begin{equation}\label{key}
		q(x_0, t_0 + \Delta t) = q(x_0, t_0) + \dfrac{v \Delta t}{\Delta x}\qty[q(x_0 - \Delta x,t_0 ) - q(x_0, t_0)]
		\end{equation}
		Our equation $ \partial_t q + \partial_x f(q) = 0 $ differs in the $ \partial_x  $ part, and hence one would have to discretize $ f(q) $ in $  x $. Using the definitions given in the problem, one can write,
		\begin{equation}\label{key}
		q^{n+1}_i = q^n_i = \dfrac{\Delta t}{\Delta x} \qty(f^n_{i - 1/2} - f^n_{i+ 1/2})
		\end{equation}
		
		\item 
	\end{enumerate}
\end{homeworkProblem}


\begin{homeworkProblem}[9]
	\textbf{Monotonicity}
\end{homeworkProblem}


\begin{homeworkProblem}[10]
	\textbf{Stiffness}
	\begin{enumerate}
		\item Given,
		\begin{equation}\label{key}
		\dv{q}{t} = - \dfrac{1}{\eta} q
		\end{equation}
		Using Euler method,
		\begin{equation}\label{eulerstiff}
		\dfrac{q^{n+1} - q^n}{\Delta t} = -\dfrac{1}{\eta} q^n \implies q^{n+1 } = \qty(1 - \dfrac{\Delta t}{\eta}) q^n
		\end{equation}
		
		\item The general solution of the ODE is $ q(t) = C e^{-t/\eta} $. If $ q(0) = 1 $, $ q(t ) = e^{-t/\eta}  \implies \lim\limits_{t \rightarrow \infty} q(t) = 0$.
		
		\item From the form in eq. \ref{eulerstiff}, we can write an ansatz solution for $ q^n $ to be,
		\begin{equation}\label{key}
		q^N = C \qty(1 - \dfrac{\Delta t}{\eta})^N \ .
		\end{equation}
		The above form of the equation suggests that $ \lim\limits_{N \rightarrow \infty}q^N = 0 \iff \Delta t \le \eta$, meaning for all other values of the timestep $ \Delta t $, the solution will blow up and go to infinity for large time.
		
		\item Applying backward difference to the original ODE gives,
		\begin{equation}\label{key}
		\dfrac{q^n - q^{n-1}}{\Delta t} = - \dfrac{1}{\eta}q^n \implies q^n = \qty(1 + \dfrac{\Delta t}{\eta})^{-1} q^{n-1}
		\end{equation}
		An ansatz solution for the above discretized form is,
		\begin{equation}\label{key}
		q^N = C \qty(1 + \dfrac{\Delta t}{\eta})^{-N} \ .
		\end{equation}
		As one can see, $ \lim\limits_{N \rightarrow \infty}q^N = 0 \qq{,} \forall \Delta t $.
	\end{enumerate}
\end{homeworkProblem}
\end{document}
